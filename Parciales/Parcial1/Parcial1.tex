\documentclass[10.5pt]{article}

% Spanish characters
\usepackage[utf8]{inputenc}
\usepackage[T1]{fontenc}
% French display
\usepackage[english,spanish]{babel}

\usepackage{lastpage}
%Esto me permite usar el comando "\pageref{LastPage}" en el footer.
\renewcommand{\baselinestretch}{1.6}
% Esto controla el interlineado o espaciado!!!
\usepackage{color}
%\newcommand{\red}[1]{{\color{red} #1}}
\newcommand{\red}[1]{{\color{black} #1}}

%Esto me permite poner hipervínculos:
%\usepackage[pdftex,
%       colorlinks=true,
%       urlcolor=blue,       % \href{...}{...} external (URL)
%       filecolor=green,     % \href{...} local file
%       linkcolor=black,       % \ref{...} and \pageref{...}
%       pdftitle={Papers by AUTHOR},
%       pdfauthor={Your Name},
%       pdfsubject={Just a test},
%       pdfkeywords={test testing testable},
%%       pagebackref,%Esto parece que pone un numerito al lado de la referencia (en la seccion de bibliografia), donde se puede clicar y te lleva al lugar del texto donde se le cita.
%       pdfpagemode=None,
%       bookmarksopen=true]{hyperref}


%The following packages are relics, but I don't want to remove them just in case:
\usepackage{amsmath}
\usepackage{array}
\usepackage{latexsym}
\usepackage{amsfonts}
\usepackage{amsthm}
\usepackage{cite}
\usepackage{setspace}
\usepackage{amssymb}
\usepackage{hyperref}

\usepackage{multicol}
\usepackage{color}
%\usepackage{minipage}

\usepackage{graphicx} % Required for including images
\graphicspath{{figures/}} % Location of the graphics files
\usepackage[font=small,labelfont=bf]{caption} % Required for specifying captions to tables and figures

%The defaults margins are huge, so I'll customize it:
\oddsidemargin  -0.0 in
\textwidth 6.5 in
\textheight 8.7 in
\addtolength{\voffset}{-1cm}

%%%%%%%%%%%%%%%%%%%%%%%% HEADER AND FOOTER %%%%%%%%%%%%%%%%%%%%
\usepackage{fancyhdr}
\pagestyle{fancy}

\fancyhead[L]{Lecci\'{o}n 1}
%\fancyhead[L]{CNRS Competition 01-04}
\fancyhead[R]{Jos\'{e} David Ruiz \'{A}lvarez}
\fancyhead[C]{}
\fancyfoot[C]{\thepage /\pageref{LastPage}}

\newlength\FHoffset
\setlength\FHoffset{0cm}

\addtolength\headwidth{2\FHoffset}
\fancyheadoffset{\FHoffset}

\newlength\FHleft
\newlength\FHright

\setlength\FHleft{1cm}
\setlength\FHright{1cm}

\thispagestyle{empty}
%%%%%%%%%%%%%%%%%%%%%%%% HEADER AND FOOTER %%%%%%%%%%%%%%%%%%%%



\begin{document}

%\begin{center}
\noindent
\begin{minipage}[b]{0.75\linewidth}
{\LARGE\bf Parcial 1}\\ %[1mm]
\large{Jos\'{e} David Ruiz \'{A}lvarez} \\
\small{\href{mailto:josed.ruiz@udea.edu.co}{josed.ruiz@udea.edu.co}} \\ %[3mm]
\normalsize{Instituto de Física, Facultad de Ciencias Exactas y Naturales} \\%[3mm]
\normalsize{\bf Universidad de Antioquia} \\[8mm]
\today %\\[4mm]
\end{minipage}%

\section{Contenido}

Los siguientes problemas son para ser desarrollado en grupo de m\'{a}ximo dos personas. 

\section{Problema 1: Obligatoriamente en C++}

Considere como ejemplo el juego de la vida con sus reglas tradicionales (ver \href{https://es.wikipedia.org/wiki/Juego_de_la_vida#El_juego}{Reglas}). Ahora, un juego de la vida de 2 especies que compiten, para ello considere una matriz cuadrada de dimensiones $n \times n$ en donde cada uno de sus elementos puede asumir dos valores 0, 1 o 2. El 1 representa una celda donde hay una célula viva de la especie 1 y el 2 representa una celda donde hay una célula viva de la especie 2, el 0 representa una celda donde no hay células vivas. Dicha matriz evoluciona temporalmente a trav\'{e}s de iteraciones de acuerdo a las siguiente reglas:

\begin{itemize}
\item Si un elemento tiene como valor 0 y tiene 3 vecinos con valor 1 y dos o más vecinos con valor 0, transforma su valor a 1.
\item Si un elemento tiene como valor 0 y tiene 4 vecinos con valor 2 y uno o más vecinos con valor 1, transforma su valor a 1.
\item Si un elemento tiene como valor 1 y tiene 2 o 3 vecinos con valor 1, conserva su valor de 1.
\item Si un elemento tiene como valor 2 y tiene 2 o 3 vecinos con valor 2, conserva su valor de 2.
\item Si un elemento tiene como valor 1 y tiene al menos 4 vecinos con valor 2 y máximo un vecino 1 vecino con valor 1. transforma su valor a 2. 
\item Si un elemento tiene como valor 2 y tiene 4 o 5 vecinos con valor 1 y máximo un vecino 2 vecino con valor 2. transforma su valor a 1.
\item Si un elemento tiene como valor 1 y tiene al menos 5 vecinos valor 1, transforma su valor a 0.
\item Si un elemento tiene como valor 2 y tiene 5 vecinos valor 2, transforma su valor a 0.
\end{itemize}

Desarrolle un script que evolucione el sistema del juego de la vida con la reglas descritas. 

\begin{itemize}
\item El script para que recibir por entrada de l\'{i}nea de comandos el valor $n$ que fija las dimensiones de la matriz. Sólo considere matrices cuadradas. (Valor=0.5)
\item Defina una clase que contenga la matriz y que tenga métodos que la hagan evolucionar de acuerdo a las reglas estipuladas. (Valor=1.5)
\item Haga evolucionar el sistema en al menos 1000 iteraciones e imprima en pantalla cada estado de forma tal que se vea evolucionar el sistema de forma estacionaria (que se sobreescriba cada uno de los elementos en cada iteración) (Valor=0.5)
\end{itemize}

{\bf Entregables problema 1}: Script en C++ más un script en Bash que compile y corra el script de C++. Sólo se pueden utilizar las librearias estándar de C, arrays y vectores.

\section{Problema 2: Obligatoriamente en C++ o Python}

Considere un sistema de dos gases que sólo interactuan entre ellos. Dichos gases ineractúan entre ellos de forma tal que la población de partículas del gas 1 $n_{1}$ y del gas 2 $n_{2}$ evolucionan con el tiempo de acuerdo a las siguientes ecuaciones diferenciales acopladas:

\begin{equation}
  n_{2}'=-n_{2}(c-d\times n_{1}); \; n_{1}'=n_{1}(a-b\times n_{2})
\end{equation} donde se entiende que $n_{i}'=\frac{dn_{i}}{dt}$ y $a,b,c,d$ son números reales.

Para un sistema de dos ecuaciones diferenciales acopladas, $y'(t)=f(t,x(t),y(t));\;x'(t)=g(t,x(t),y(t))$ podemos generalizar el método de Euler usando las siguientes ecuaciones:
\begin{equation}
y_{n+1}=y_{n}+hf(t_{n},x_{n},y_{n});\;x_{n+1}=x_{n}+hg(t_{n},x_{n},y_{n})
\end{equation} con $h=(b-a)/N=t_{n+1}-t_{n}$, $x_{n}=x(t_{n})$ y $y_{n}=y(t_{n})$.

Con definiciones similiares se puede también generalizar el método de Runge-Kutta de orden 4.

Solucione cada uno de los siguientes problemas:

\begin{itemize}
\item Implemente (escriba un script) el método de Euler y de RK4 para el sistema de dos ecuaciones diferenciales ordinarias acopladas. (Valor=0.5)
\item Utilice ambos métodos y la librería odeint de python para solucionar el sistema de ecuaciones descrito en la ecuación (1) con $a=b=c=d=1$ y condiciones iniciales $n_{1}(0)=1.50,\;n_{2}(0)=1.00$. Encuentre la solución al sistema para $t=[0,12]$. Grafique las soluciones en los planos $(t,n_{1})$, $(t,n_{2})$ y $(n_{1},n_{2})$ (Valor=1.0)
\item Utilizando la solución encontrada con el método odeint como si fuese la solución exacta muestre si las soluciones encontradas con el método de Euler y RK4 son o no convergentes. (Valor=1.0)
\end{itemize}

{\bf Entregables problema 2}: Scripts completos, gráficos solicitados. Script en Bash que ejecute los scripts desarrollados.

{\bf Sobrepuntaje global}: Se dará un sobrepuntaje global de 0.5 si TODOS los scripts de los dos problemas están exhaustivamente comentados y explicados.

{\bf M\'{e}todo de entrega}: Pull request al repositorio central del curso adentro de la carpeta llamada ``Parcial1'' con todos los entregables para solo uno de los estudiantes en la carpeta de su número de cédula respectivo. Debe además incluir en el pull request los nombres y números de cédulas de los dos integrantes del grupo.

\end{document}

%%% Local Variables:
%%%   mode: latex
%%%   mode: flyspell
%%%   ispell-local-dictionary: "spanish"
%%% End:
